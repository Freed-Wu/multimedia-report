\documentclass[../main]{subfiles}
\begin{document}

\begin{acknowledgement}
  本文主题之所以选择 NLP ,基于2个考虑。

  \begin{itemize}
    \item 我之前在一次数模比赛中接触过 NLP ,当时的题目是根据亚马逊的评论数据
      判断商品的好坏,问题的关键在与分词,判断词感情色彩的好坏,再以此来打分
      。方法是一个朴素贝叶斯网络。在对此有些了解之后,对这方面历史有些好奇,
      借此机会好好搜索了一下。
    \item\href{https://www.zhihu.com/question/285719774/answer/1020515592}{知
      乎}上的这个回答给出了分词后利用词频判断书籍作者的方案。我本来想提一下这
      篇文章,但仔细看了后我觉得他可能有意隐瞒了对自己观点不利的数据,所以
      没有在正文中提。从查到的结果来看,关于\setquotestyle{book}\enquote{红楼
      梦}\setquotestyle{zh}作者这个问题的争论还是很大的。
  \end{itemize}

  在这门课的学习过程中,我首先要感谢的是我的指导老师康其桔老师,康老师严谨的
  研究态度深深的影响着我。康老师不仅以严谨的治学态度给我以后的学习研究树立了
  典范,更使我对人生有了新的领悟。感谢康老师的悉心指导!

  本文开源于
  \href{https://github.com/Freed-Wu/multimedia-report}{Freed-Wu/multimedia-report},
  欢迎留言指正!
\end{acknowledgement}

\end{document}

