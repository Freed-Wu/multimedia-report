\documentclass[../main]{subfiles}
\begin{document}

\chapter{典型应用}%
\label{cha:method}

\section{机器翻译}%
\label{sec:translate}

\begin{definition}[机器翻译]
  由计算机程序将文字或演说从一种自然语言翻译成另一种自然语言。
\end{definition}

借由使用语料库的技术,可达成更加复杂的自动翻译,包含可更佳的处理不同的文法结
构、辞汇辨识、惯用语的对应等。

目前的机器翻译软件通常可允许针对特定领域或是专业领域(例如天气预报)来加以客
制化,目的在于将辞汇的取代缩小于该特定领域的专有名词上,以借此改进翻译的结果
。这样的技术适合针对一些使用较正规或是较制式化陈述方式的领域。例如政府机关公
文或是法律相关文件,这类型的文句通常比一般的文句更加正式与制式化,其机器翻译
的结果通常比日常对话等非正式场合所使用语言的翻译结果更加符合语法。

\begin{description}
  \item[辞典法]机器翻译可利用辞典的词汇作翻译。因为这种翻译是“字对字”的,所以
    通常各字之间在意思上都没有任何关联。这种机器翻译法最适用于具有冗长的词语
    列表(意即非完整的句子)。例如产品型录的翻译。一般机器翻译都会采用一种人
    工国际语言作为中介语言,如图~\ref{fig:translate}。
  \item[范例法]基于实例的翻译方法。基本思路是电脑模拟大量翻译实例(翻译语料库
    ),进行有效替换的翻译策略。因此该方法依赖于翻译语料库的质量、规模和覆盖
    面。如果有完全一样的例句,则直接采用范例的译文;如果有多个相似的例句,则
    自动模拟相似度最高的译文,只需翻译不同部分即可;如果没有相似的译文,则必
    须进行基于统计或规则的方法进行翻译。根据乔姆斯基的转换生成语法而言,这种
    方法永远也无法赶上人的语言的变化。因此,这种方法算是比较笨的方法,类似于
    字典,我们可以从中查到有用的字词,甚至短语,但写出什么东西,却是字典无法
    实现的。因此这种方法有一定的实用性,但局限性也显而易见。
  \item[统计法]是目前非限定领域机器翻译中,性能较佳的一种方法。统计机器翻译的
    基本思想是通过对大量的平行语料进行统计分析,构建统计翻译模型,进而使用此
    模型进行翻译。从早期基于词的机器翻译已经过渡到基于短语的翻译,并正在融合
    句法信息,以进一步提高翻译的精确性。

    统计机器翻译的首要任务是为语言的产生构造某种合理的统计模型,并在此统计模
    型基础上,定义要估计的模型参数,并设计参数估计算法。早期的基于词的统计机
    器翻译采用的是噪声信道模型,采用最大似然准则进行无监督训练,而近年来常用
    的基于短语的统计机器翻译则采用区分性训练方法,一般来说需要参考语料进行有
    监督训练。贝氏模型(Bayesian Model)也是一种机器翻译方法。

    近年来在语言服务产业掀起波澜的神经机器翻译就是利用巨大的人工神经网络计算
    一连串字词的几率以产生文意精确的翻译。将在未来数年持续改变翻译及语言在地
    化产业。在投入大量翻译资料集(data sets)以训练人工智能和机器学习模组后,
    神经机器翻译的品质已大幅改善。更重要的是,当神经机器翻译与人工编修搭配,
    无论在技术还是文化层面,更能达到ㄧ流的译文品质。因此,神经翻译与人工编修
    搭配需求庞大。
\end{description}

\begin{figure}[htbp]
  \centering
  \includegraphics[
    width = 0.8\linewidth,
  ]{translate}
  \caption{机器翻译}%
  \label{fig:translate}
\end{figure}

\section{智能交互}%
\label{sec:assistant}

模拟人类来和人类进行交流。语音助手即是其最常见的代表。

图灵最早提出图灵测试来验证人工智能。但关于人类能否造出真正的人工智能,学术上
一直争论不止。最出名的一个理论是翻译房子:把一个不会中文的外国人关在一个房子
里,里面只有一本词典,当有人从窗户递过一条中文纸条时,该外国人利用词典将回答
的中文打印在纸条上递出去,造成一种屋内人会中文的假象,实际上他并不会中文。有
批评者认为指望这种方式来判断与人交互的软件拥有智能是不对的。

但实际上另一种理论叫鸭子测试:如何判别一个事物是鸭子?回答是:如果它能游泳,
能像鸭子一样叫,长得像只鸟,那么它就是鸭子。图灵测试很显然也是鸭子测试的一种
。在程序设计中,鸭子测试已经被广泛采用,比如早期的语言以强类型为主,C 语言声
明一个变量必须指出该变量是哪一种类型。但后期的语言以弱类型为主,Python 等都
运行用户直接声明一个没有类型的变量,即鸭子变量,只有当程序运行时鸭子变量才会
被真正作为某种具体类型实现。

所以,无论跟我们聊天的语音助手到底有没有智能,只要它能骗过大多数的人,他就
可以被认定为有智能。就像是一只能游泳,能像鸭子一样叫的鸟是鸭子一样。

但相应的,也有反鸭子测试:如果它能游泳,能像鸭子一样叫,长得像只鸟,但它一旦
离开电池就无法运动,那么它就一定不是鸭子。所谓的语音助手,终究只是一只像鸭子
但不是鸭子的事物罢了。

抛开关于人工智能的争论不谈,语音助手的发展还是为人们的生活带来了许多便利。
至于它能不能达到智能的高度,答案就留给未来去见分晓吧。

\section{作者考证}%
\label{sec:author}

利用 NLP 技术将书籍转变为计算机可以理解的数据,经过数理统计分析判断书籍的作者
。最出名的例子是判断\setquotestyle{book}\enquote{红楼梦}的作者。不过值得指出
的是,\enquote{红楼梦}\setquotestyle{zh}的作者至今悬而未觉。权威的说法也只是
曹雪芹着,无名氏续。

1981年,陈炳藻通过对红楼梦的数理统计,得出全120回皆为曹雪芹原作的结论。但学术
界依旧没有对此达成普遍共识,而赵冈更是同样通过数理统计得出完全相反的意见。此
后不同学者用数理统计方法做了研究,一些研究认为前80回与后40回用词风格相同,而
另一些研究表明前80回与后40回用词风格不同,但是,数理统计也存在以下问题:检验
项的选择上有主观倾向,统计结果出现了一些例外无法解释。有学者使用某些数理统计
方法研究其他作家的不同文学作品,发现同一作者同一作品的不同章回有可能被分为不
同类,不同作者的作品又有可能被分为一类,因此认为,某些数理统计方法并不能成为
确证。对于风格相同的结果,持前后不同作者说的人认为这是后40回作者模仿前80回作
者的遣词用句所导致的,并非一定由同一作者所写。对于风格不同的结果,持前后基本
为同一作者说的人认为,据《红楼梦》第一回所说,作者写作时间很长,前后增删改数
次,这也可能造成同一部小说前后用词风格差异,而并非一定是不同作者写成。此外,
如果程高二人对后四十回的原稿进行了增补这一说法是真的,那也有可能造成前后用词
风格不一致,但若是这样,后四十回仍会包含原稿内容而不是完全由程高二人虚构。因
此,即使使用了数理统计方法,作者问题仍然悬而未决。

\end{document}

